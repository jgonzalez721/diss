% Sample Dissertation, Thesis, or Document %
%            for use with the              %
%  University of Arizona Thesis Class,     %
%               uathesis.cls               %
%------------------------------------------%

% This is an adapted version of J.J. Charfman's template which can be
% found here: https://da.overleaf.com/latex/templates/university-of-arizona-astronomy-thesis-template/tsfqcgnfmjcx
% This template is adapted for linguistics. The uncommented line
% below will produce a Dissertation, the others would produce a Thesis
% or a Document.  There are other options available to you like turning
% on the copyright statement and replacing the year on the title page
% with a "generated on" stamp (handy for early drafts).  To find out
% what the available options are, take a look into the uathesis.cls
% file and look for the \DeclareOption commands near the top of that
% file.
% There are five copyright options.  Copyright, no copyright, and three
% different Creative Commons licences.  Use the one you want (If you go
% Creative Commons, I (DM) think the CC-BY-ND makes the most sense)  See
% uathesis.cls for the reason why the non-commercial licenses are not
% included.
\documentclass[dissertation,xcolor={table,dvipnames}]{uathesis} % we require tikz, which requires xcolor, if you want to load these packages with options, then do it as seen here.
%\documentclass[dissertation,copyright]{uathesis}
%\documentclass[dissertation,CC-BY]{uathesis}
%\documentclass[dissertation,CC-BY-SA]{uathesis}
%\documentclass[dissertation,CC-BY-ND]{uathesis}
%\documentclass[thesis]{uathesis}
%\documentclass[document]{uathesis}

% Package Usage
% These are the packages that I recommend


%\usepackage{color} %broader color range and color options, allows you to define your
                    % own colors as RBG vectors


\usepackage{graphicx} % allows insertion of pictures etc



\usepackage[utf8]{inputenc} %allows you to directly input utf-8 characters
                            % this means you can just type Árüz instead of \'Ar\"uz

                            
% \usepackage{natbib}
% \setcitestyle{semicolon,aysep={},yysep={,},notesep={:}} % natbib is available on most systems, and is
					% terribly handy.
					% We are using biber for this template, but leaving natbib in as an option
                    % if you want to use natbib instead of biber/biblatex, you need to 
                    % comment out the next package and the % addbibresource params
\usepackage[backend=biber,
                style=unified, % we are using the LSA unified style sheet here
                maxcitenames=2,
                maxbibnames=99,
                natbib]{bib latex}
\addbibresource{bibliography.bib}


\usepackage{tipa} %enables an IPA environment

\usepackage{qtree} %allows simple trees
\usepackage[linguistics]{forest} %allows more complex trees


\usepackage{fancyhdr} % cosmetics for better headings



% \usepackage{times}  %enable if you want to use a times font
\usepackage{amsmath} %this allows the use of \eqref{}, eqref will automatically put 
                    % brackets around your reference, which is great for referencing
                    % gb4e examples
\usepackage[hidelinks]{hyperref} %allows to use references as links [hidelinks] makes sure there are no borders around the link, gives me \autoref


\usepackage{lscape}			% Used for making fitting large tables in by putting them landscape
\usepackage[tableposition=t]{caption} % (makes table captions normal size when font set smaller)


% \usepackage{enumitem} % allows advanced controls of tables and example numbering
		

% If you are using hyper-ref (recommended), this command must go after all 
% other package inclusions (from the hyperref package documentation).
% The purpose of hyperref is to make the PDF created extensively
% cross-referenced.
%allows to use references as links [hidelinks] makes sure there are no borders around the link, gives us \autoref
\usepackage[hidelinks]{hyperref}

\usepackage{gb4e} %manages example numbering, has to be last package in pre-amble, will break document otherwise

% this file can host your custom commands so they do not clutter your mainfile

% for documentation on LaTeX commands see: https://www.overleaf.com/learn/latex/Commands#Simple_commands


%%% The following allows dagger footnote on chapter titles
%%% with no number. Useful for designating previously 
%%% published work.
\newcounter{daggerfootnote}
\newcommand*{\daggerfootnote}[1]{%
    \setcounter{daggerfootnote}{\value{footnote}}%
    \renewcommand*{\thefootnote}{\fnsymbol{footnote}}%
    \footnote[2]{#1}%
    \setcounter{footnote}{\value{daggerfootnote}}%
    \renewcommand*{\thefootnote}{\arabic{footnote}}%
    }


 % we are hosting all the custom commands we are writing in a seperate file

% Set up some values.
\completetitle{}
\fullname{}			% Grad college wants your full name here.
\degreename{Doctor of Philosophy}	% Title of your degree.
\degreemajor{Linguistics} % Degree major

\counterwithout{footnote}{chapter} %disable this if you want your footnotes to reset their counter for each chapter

% the inlinetitle below will be used on the graduate college signature page if you uncomment it. 
% This is useful if your completetitle uses a linebreak or other command that would break an inline use of that title. 
% If you leave this commented it will do nothing. Uncomment and give it a value to activate. The change will be visible on page 2.

%\inlinetitle{}

\begin{document}

% tocdepth defines how many layers of sections/subsections are displayed in your table of content. The hierarchy works like this:
% Chapters (1)> Sections (2) > Subsections (3) > SubSubsections (4) > Paragraphs (5)
% Default = 4
%\setcounter{tocdepth}{5} 
% secnumdepth does the same work, except it numbers your sections. Chapters are automatically numbered, so setting this to 4 will give you numbered paragraphs. Default = 3
%\setcounter{secnumdepth}{4}



% Set up the title page
\maketitlepage
{DEPARTMENT OF LINGUISTICS}	% Title of your department.
{2019}							

% Insert the approval form.  Note that for electronic submission
% of your Ph. D. dissertation, you must bring *two* copies of the
% approval page to your final defense.  These must be signed by
% the committee.  Make two photocopies: one for the department
% and the other for your records.  Then, bring the two signed 
% originals to the graduate college when you submit the 
% final version of the dissertation to the University of Arizona.
% if you want to change the number of lines displayed, you need to do this in the uathesis.cls class, check the \approval command
\approval
{}		% Defense Date	
{}	% Dissertation Director
{}	% 1st committee member
{}		% 2nd committee member
{}		% 3rd committee member
{}		% 4th committee member
{}		% 5th committee member
{}		% 6th committee member

% if you want to load your approval with a UA watermark, use this command, which uses the same syntax as the one above:
% \approvalWithWM{}{}{}{}{}{}{}{}

% Include the ``Statement by Author'' for Dissertations
\statementbyauthor
% If this is a Thesis, use the following form, with your thesis director's
% name and title in the square brackets like so (you should also omit the 
% approval form insertion above):
%\statementbyauthor[Jane M. Doe\\Professor of Chemistry]

% Include the ``Acknowledgements''
\incacknowledgements{acknowledgements}

% Include the ``Dedication''
\incdedication{dedication}

% Create a ``Table of Contents''
\tableofcontents

% Create a ``List of Figures''
\listoffigures

% Create a ``List of Tables''
\listoftables

% Include the ``Abstract''
\incabstract{abstract}

% Include the various chapters
\chapter[Introduction]% the [square brackets] define the name of this in the TOC
{Introduction\daggerfootnote{This chapter has been published previously as \citet{Chimsky2022}.}}\label{chap:Intro} %we define a label for this chapter



Here are different cite commands as demonstrated in a \emph{gb4e} list:

\begin{exe}
    \ex \textbackslash cite : \cite{chomsky-why-2016}
    \ex now here is a sublist \begin{xlista}
    \ex \textbackslash citep : \citep{chomsky-why-2016}
    \ex \textbackslash citet{} : \citet{chomsky-why-2016}
    \end{xlista}
    \ex\label{item} now there is another numbered item
\end{exe}
And now I reference that item with \textbackslash eqref: \eqref{item}
%\begin{figure}
%\centering
%\includegraphics[angle=0,width=\columnwidth]{fig1.pdf}
%\caption[]{}
%\label{fig1}
%\end{figure}
\chapter[Chapter 2]{TitleofChapter}

Let's make a figure out of a tree and position it right under our text:

\begin{figure}[h] %to understand figure positioning, please refer to: 
% https://www.overleaf.com/learn/latex/Inserting_Images
    \centering
    \begin{forest}
    [NP [DP] [N] ]
    \end{forest}
    \caption[A very special figure]{Caption}
    \label{fig:mylabel}
\end{figure}

And here is a table which we position at the bottom:
%%%%%%%%% some custom definitions first
\newcolumntype{s}{>{\columncolor[HTML]{AAACED}} p{3cm}} %here we define a custom column
% type, called 's', this will have the color as defined via HTML and a width of 3cm
\arrayrulecolor[HTML]{DB5800}
\colorlet{mycolor}{red!30!blue!50} % here we define our own color, using red  
                                    % at 30% opacity and blue at 50%
%%%%%%%%% https://www.overleaf.com/learn/latex/Tables
\begin{table}[b!]
\begin{tabular}{ |s|p{3cm}|p{3cm}| }
\hline
\rowcolor{lightgray} \multicolumn{3}{|c|}{Country List} \\
\hline
Country Name or Area Name \cellcolor{mycolor}
& ISO ALPHA 2 Code &ISO ALPHA 3 \\
\hline
Afghanistan & AF &AFG \\
\rowcolor{gray}
Aland Islands & AX & ALA \\
Albania   &AL & ALB \\
Algeria  &DZ & DZA \\
American Samoa & AS & ASM \\
Andorra & AD & \cellcolor[HTML]{AA0044} AND    \\
Angola & AO & AGO \\
\hline
\end{tabular}
\caption[the table's list reference in the list of tables]{A table with some color, positioned at the bottom of a page}
\end{table}

\chapter[ChapterShortTitle]{ChapterTitle}

Now let's walk through some \textbackslash ref commands:

\begin{exe}
    \ex \textbackslash ref: \ref{chap:Intro}
    \ex \textbackslash autoref: \autoref{chap:Intro}
    \ex \textbackslash eqref: \eqref{chap:Intro}
    \ex \textbackslash nameref: \nameref{chap:Intro}
    
\end{exe}


% Include the various appendices
\appendix
\chapter{Extra Stuff}

%\begin{figure}
%\centering
%\includegraphics[angle=0,width=\columnwidth]{fig1.pdf}
%\caption[]{}
%\label{fig1}
%\end{figure}


% Switch the spacing to single-spaced for the references
\renewcommand{\baselinestretch}{1}		% changing the value
\small\normalsize						% switch size to make the value take

% Use the below if using natbib
% this is where your natbib stylesheet is referenced
%\bibliographystyle{sp-lsa.bst}	% S\&P bibliography style
% S\&P, an LSA publication, meets the LSA guidelines.  
% if you wish to use the UA citation style instead, call "uabibnat.bst" in the \bibliographystyle
%\bibliography{bibliography}

% use this for biblatex/biber
\printbibliography

\end{document}
